\documentclass[12pt]{article}
 
\usepackage[margin=1in]{geometry} 
\usepackage{amsmath,amsthm,amssymb}
\usepackage{enumitem}
\usepackage{graphicx}
\usepackage{multirow}
\usepackage{siunitx}
 
\begin{document}

 
\title{Internet of Things - Smart Home Edition}%replace X with the appropriate number
\author{Noman Bashir, Shubham Mukherjee}
\maketitle

\section{Latency Results}

We measure the latency of different actions performed by the system. All the results are presented 
below: 

\noindent
\textbf{Cache Results:}
The first one is the 
time taken to process a request with and without cache. We evaluate this by taking a sample 
user activity and measuring the total time taken by the sequence of actions with and without 
cache. For example, in order to process the user activity profile 1 the times taken are:

\begin{itemize}
	\item Time taken with caching: 11.7785770893
	\item Time taken without caching: 12.8137729168
\end{itemize}

There are total of three cache misses without cache and the total time taken by them all is 
1.035193 second which results in an average of 0.345 second per cache miss. 

\noindent
\textbf{Failure Results:} We configure the wait time for the heartbeat as the heartbeat period + extra time
to account for the message delays. 

\begin{itemize}
	\item Time taken to detect failure: heartbeat period + extra time + arbitrary time. The evaluator can configure these
	to any desired values. However, extra time should be higher than the typical message propagation 
	delays which is 10ms. If heartbeat period is 5second, extra time is 1seconds, teh total time to detect failure is around 8second.
	\item Time taken taken to recover from a failure: detecting failure time plus extra time of 5second. For the 
	values given above, it takes around 14seconds on average to recover from the crash. 
\end{itemize}




\section{Test Case 1}
This test case aims to show that our mechanism dynamically balances the 
load across the gateways. 

\begin{itemize}
	\item We run the registerProcesses.py script. It asks the user to input the 
	number of devices to be registered between (1-6). 
	\item After user enters the no. of devices, the script registers the devices 
	with the two gateways. The device ID and the gateway with which it is 
	registered is printed. 
	\item It can be seen from the command line output that the devices are 
	evenly divided across gateways. However, in case of odd number of devices, 
	one gateways gets one device more than the other. 
	\item the results can be further verified by running the Pyro4-nsc list for 
	both the servers. 
\end{itemize}


\section{Test Case 2}

The purpose of this test case is to demonstrate the implementation 
of consistency mechanism. In order to perform this test, the evaluator 
needs to uncomment the 
\textbf{test\_consistency} test case in run.py.

\begin{itemize}
	\item After completing the registration process, the user performs some 
	actions on different devices distributed across both gateways. 
	\item We then retrieve all the values from the database and print them  to show that 
	the state of databases in both cases is same. 
\end{itemize}


\section{Test Case 3}

This test case targets to demonstrate the working of caching mechanism in 
our system. For this test case, we request the evaluators to change the 
value of \textbf{constant \textit{cacheSize} in config filec}. They should change the cache 
size from 0 to 3. We provide the reference output for all the cache cases. 
You can enable any one test available in the run.py file and evaluate its 
performance with different cache sizes. 

\begin{itemize}
	\item When the cache size is set to 0, all of the requests for database will result in cache miss
	as there won't be any value stored in the cache. 
	\item In case of cache size 1, there will be some cache hits and some misses. This will happen 
	as we need to last two entries for event order logic in some cases. 
	\item For cache size 2 or greater than 2, every request for database will result in a cache hit. 
	This is because our event order logic requires only last two entries.
\end{itemize}

\section{Test Case 4}

This test case aims to evaluate the fault tolerance mechanism of our project. In order to 
evaluate this test case, the evaluators need to crash one name server. 
\begin{itemize}
	\item The evaluator can enable any of the three user activity test profiles in the 
	run.py file. 
	\item The evaluator will have to crash one name server at any point in time they desire. 
	\item The devices will register themselves with the other gateway after some time. 
	\item Sample output is provided in the test directory.
\end{itemize}




\end{document}
